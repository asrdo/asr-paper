\section{Similarities and Overlapping Themes Across The Papers}
\subsection{Historical Evolution}
\begin{itemize}
    \item Several papers discuss the historical evolution of ASR, starting from early analog devices (e.g., "Radio Rex") to statistical methods like Hidden Markov Models (HMMs), and the rise of deep neural networks (DNNs).
    \item Common acknowledgment of Moore’s Law and advances in compute power as a driving force behind modern ASR systems.
\end{itemize}

\subsection{Challenges in ASR}
\begin{itemize}
    \item Handling noisy or accented speech.
    \item Limited datasets for specific languages (e.g., Arabic Dialectal ASR).
    \item Computational efficiency and reducing word error rates (WER).
\end{itemize}

\subsection{Future Directions}
\begin{itemize}
    \item Incorporating cognitive-inspired architectures.
    \item Exploring prosodic cues for improved recognition.
    \item Enhancing usability for downstream applications, such as transcription and voice assistants.
\end{itemize}

\section{Similar Points, Methods, Algorithms, and Techniques}
\subsection{Hidden Markov Models (HMMs)}
\begin{itemize}
    \item Central to traditional ASR methods for managing variability and speech modeling (e.g., Xuedong Huang, 2014; O'Shaughnessy, 2023).
\end{itemize}

\subsection{Deep Neural Networks (DNNs)}
\begin{itemize}
    \item Widely adopted in modern ASR systems for acoustic modeling and improving speech accuracy (Douglas O'Shaughnessy, 2023; Hannun, 2021).
\end{itemize}

\subsection{Acoustic-Phonetic Approach}
\begin{itemize}
    \item Focuses on decoding speech based on phonetic units, as highlighted in early reviews (e.g., Reddy, 1976; Anusuya \& Katti, 2010).
\end{itemize}

\subsection{Feature Extraction Techniques}
Shared algorithms include:
\begin{itemize}
    \item MFCC for spectral representation.
    \item Linear Predictive Coding (LPC).
    \item Formant tracking for speech structure analysis.
\end{itemize}

\subsection{Language Models}
Discussed approaches include:
\begin{itemize}
    \item N-gram modeling (for statistical prediction of word sequences).
    \item Decision-tree clustering (for acoustic context).
\end{itemize}

\subsection{Software Frameworks}
\begin{itemize}
    \item Mention of open-source tools like Kaldi, TensorFlow, and PyTorch as foundational frameworks for ASR development.
\end{itemize}