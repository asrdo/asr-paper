
\documentclass[12pt]{article}
\usepackage[margin=1in]{geometry}
\usepackage{times}
\date{}
\begin{document}

\section*{Speech Recognition by Machine: A Review}
This article offers a comprehensive survey of the technological developments and foundational concepts in Automatic Speech Recognition (ASR) spanning six decades of research. It explores three major approaches in the field: the Acoustic-Phonetic approach, which attempts to decode speech by analyzing phonetic units; the Pattern Recognition approach, which relies on statistical models like Hidden Markov Models (HMMs) and techniques like Dynamic Time Warping (DTW); and the Artificial Intelligence approach, ...

The article also categorizes ASR systems based on the type of speech they process (e.g., isolated words, connected speech, continuous, and spontaneous speech) and their real-world applications in fields like telecommunications, education, healthcare, and military systems. It outlines key performance factors including vocabulary size, noise robustness, speaker independence, and system adaptability. Important technical components such as Mel Frequency Cepstral Coefficients (MFCC), Linear Predictive Coding ...

Lastly, the review provides a historical perspective on the evolution of ASR technologies from the 1920s to the 1990s. From early analog devices like "Radio Rex" to the introduction of neural networks and expert systems, the paper shows how technological advancements shaped modern ASR. It concludes by identifying open research areas such as improving spontaneous speech recognition and system personalization, suggesting opportunities for future innovation in the field.

\section*{The History of Speech Recognition to the Year 2030}
Awni Hannun's article offers a forward-looking narrative on the state and future of ASR. It reviews major breakthroughs between 2010 and 2020—highlighting the impact of deep learning, massive transcribed datasets, and GPU acceleration. These advances drastically reduced Word Error Rates (WER) and paved the way for robust voice assistants and far-field applications. The transition from hybrid HMM-DNN systems to end-to-end neural architectures, including Deep Speech and transducer-based models, marked a m...

Looking ahead, Hannun predicts ASR systems will prioritize usability and integration over benchmark metrics. Techniques such as self- and semi-supervised learning, on-device inference, and model compression through distillation and sparsity will drive personalization and privacy. He expects ASR models to adapt dynamically to user environments, accents, and speech disorders, enabling seamless, always-available speech recognition even offline.

In closing, Hannun acknowledges the growing centralization of ASR research within large tech firms, raising concerns about limited access for academic communities. Despite this, the article maintains an optimistic view of future ASR developments, especially in terms of accessibility, personalization, and the integration of richer semantic understanding across applications.

\end{document}
